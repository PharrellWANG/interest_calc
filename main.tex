\documentclass[a4paper]{article}

%% Language and font encodings
\usepackage[english]{babel}
\usepackage[utf8]{inputenc}
\usepackage[T1]{fontenc}

%% Sets page size and margins
\usepackage[a4paper,top=3cm,bottom=2cm,left=3cm,right=3cm,marginparwidth=1.75cm]{geometry}

%% Useful packages
\usepackage{amsmath}
\usepackage{graphicx}
\usepackage[colorinlistoftodos]{todonotes}
\usepackage[colorlinks=true, allcolors=blue]{hyperref}
\usepackage[backend=bibtex,style=ieee,natbib=true]{biblatex}
\addbibresource{sample.bib} % The filename of the bibliography
\hypersetup{
colorlinks=false, %set true if you want colored links
linktoc=all,     %set to all if you want both sections and subsections linked
linkcolor=black,  %choose some color if you want links to stand out
}
\title{Zwap Interest Calculation}
\author{Pharrell.zx}

\begin{document}
\maketitle

\begin{abstract}
This document specifies how to calculate below items for short-term loan (a.k.a fun fund loan).
Source~\cite{RN001}.
\begin{itemize}
      \item early settlement amount
      \item total repay amount when overdue within 1 month
\end{itemize}
\end{abstract}

\section{Early settlement amount}
Let us denote
\begin{itemize}
      \item \textbf{interest rate per day} as \(x\),
      \item \textbf{borrower interest rate} as \(b\), it would be 0.05 if no discount, else if the discount is 0.9, it would be 0.045.
      \item \textbf{repay date} as \(d_c\)
      \item \textbf{draw down date} as \(d_d\)
      \item \textbf{overdue days} as \(d_o\)
      \item \textbf{total principal (the disbursed amount)} as \(p\)
      \item \textbf{early settlement amount} as \(A_{es}\)
\end{itemize}

Then the early settlement amount is: 

\[A_{es} = p\times(1+x\times d_o)\]

here

\[d_o = d_c - d_d\]
\[x =\frac{b}{90}\]

\section{Repay amount when overdue within 1 month}

Let us denote
\begin{itemize}
      \item \textbf{interest rate per day} as \(x\),
      \item \textbf{borrower interest rate} as \(b\), it would be 0.05 if no discount, else if the discount is 0.9, it would be 0.045.
      \item \textbf{repay date} as \(d_c\)
      \item \textbf{due date} as \(d_d\)
      \item \textbf{overdue days} as \(d_o\)
      \item \textbf{total principal (the disbursed amount)} as \(p\)
      \item \textbf{amount to pay} as \(A_{t}\)
\end{itemize}

Then the amount is: 
\[A_{t} = (p\times(1+b))\times(1+x\times d_o)\]

here

\[d_o = d_c - d_d\]
\[x =\frac{b}{90}\]

% Source~\cite{greenwade93}

% \bibliographystyle{alpha}
% \bibliography{sample}
\printbibliography[heading=bibintoc]
\end{document}